\documentclass[12pt,a4paper]{article}

% Paquetes recomendados
\usepackage[utf8]{inputenc}   % Codificación de caracteres
\usepackage[T1]{fontenc}      
\usepackage[spanish]{babel}   % Idioma en español
\usepackage{setspace}         % Espaciado
\usepackage{geometry}         % Márgenes
\usepackage{parskip}          % Espaciado entre párrafos
\usepackage{hyperref} 

% Configuración de la página
\geometry{top=2.5cm, bottom=2.5cm, left=3cm, right=3cm}
\setstretch{1.2} % Interlineado

% Datos del documento
\title{\textbf{Alcance}}
\author{Emiliano Guardabassi}
\date{} % Sin fecha

\begin{document}

\maketitle

\section*{Alcance del Proyecto}
El objetivo de este proyecto es desarrollar un sistema que permita a los usuarios jugar una versión del juego de mesa
\textit{Agatha Christie's Death on the cards} como aplicación web. Pudiendo soportar múltiples partidas multijugador en tiempo real.
Las reglas están detalladas \href{https://drive.google.com/file/d/1FV6UZLrJWPkj3h-fCvlWORsK6dP490Zc/view}{aquí}.

La aplicación web recibirá al usuario pidiendole una serie de datos para
identificarlo durante la partida y establecer el orden de turnos acorde a
las reglas. Estos datos serán:
\begin{enumerate}
    \item Avatar
    \item Nombre
    \item Fecha de cumpleaños
\end{enumerate}
Dichos datos no serán persistentes, es decir, no se almacenarán en una base de
datos, sino que se mantendrán únicamente durante la sesión del usuario en la
aplicación web. En caso de que el usuario desee volver a jugar, deberá ingresar
nuevamente sus datos.

Posteriormente, el usuario podrá optar por crear una nueva partida y ser el 
\textbf{anfitrión} ó unirse a una ya existente. En caso de optar por crear 
una nueva, el usuario deberá establecer una serie de parámetros para la misma, 
los cuales serán:
\begin{enumerate}
    \item Nombre de la partida
    \item Cantidad mínima de jugadores (2 - 6)
    \item Cantidad máxima de jugadores (2 - 6)
    \item Contraseña (opcional)
    \item Temporizador de turno (10s - 120s ó 60s por defecto)
\end{enumerate}
Por otro lado, si el usuario decide unirse a una partida ya existente, deberá
ingresar el nombre de la partida y, en caso de que la misma esté protegida por
una contraseña, deberá ingresarla también. La aplicación web mostrará una lista
de las partidas disponibles, incluyendo el nombre de la partida, la cantidad
de jugadores actuales y la cantidad máxima de jugadores permitidos.

Después de que un usuario crea o se une a una partida, será dirigido a una sala
de espera. Solamente el \textbf{anfitrión} puede iniciar la partida. Es necesario
que la cantidad de jugadores presentes alcance al menos el mínimo requerido
(2 ó el establecido en la configuración de partida) para poder iniciar. Los
participantes tienen la posibilidad de abandonar la sala de espera en cualquier 
momento. Si el anfitrión decide salir de la sala, la partida se eliminará automáticamente.

Cuando se inicia la partida, los jugadores son redirigidos automáticamente 
al entorno de juego, donde se desarrollará la partida. Los jugadores participarán
por turnos acorde a las reglas, cada uno limitado por el temporizador configurado. 
Durante la partida, podrán interactuar mediante un \textbf{chat de texto}, 
jugar cartas con distintos efectos según las reglas, y visualizar el estado 
actual del juego. Los jugadores no pueden abandonar la partida en curso. 
En caso de que un jugador no realice jugadas por 2 turnos consecutivos, se le 
considerará \textbf{``inactivo''} y se le acortará el temporizador de turno a 
10 segundos hasta que demuestre signos de actividad nuevamente, cuando se 
restablecerá su temporizador original.


\vspace{2cm}
\begin{flushright}
\textbf{Emiliano Guardabassi}
\end{flushright}

\end{document}
