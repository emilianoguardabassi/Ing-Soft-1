\documentclass[12pt,a4paper]{article}

% Paquetes útiles
\usepackage[utf8]{inputenc}
\usepackage[T1]{fontenc}
\usepackage[spanish]{babel}
\usepackage{enumitem}   % Mejor control de listas
\usepackage{geometry}   % Márgenes
\usepackage{setspace}   % Espaciado
\usepackage{parskip}    % Espacio entre párrafos
\usepackage{hyperref} 

% Configuración
\geometry{top=2.5cm, bottom=2.5cm, left=3cm, right=3cm}
\setstretch{1.2}

% Definir un entorno para Casos de Uso
% \newcounter{caso}
% \newenvironment{casodeuso}[1]{
%   \refstepcounter{caso}
%   \section*{Caso de Uso \#\thecaso: #1}
% }{}

\newenvironment{casodeuso}[2]{%
  \section*{Caso de Uso \##1: #2}%
  \addcontentsline{toc}{section}{Caso de Uso \##1: #2}%
  \label{cu:#1}%
}{}

\begin{document}

\title{Casos de uso pedidos}
\author{Emiliano Guardabassi}
\date{}
\maketitle


\textbf{Reglas del juego:} \href{https://drive.google.com/file/d/1FV6UZLrJWPkj3h-fCvlWORsK6dP490Zc/view}{\textbf{Link}}

\begin{casodeuso}{4}{Iniciar partida}
\textbf{Actor Primario:} Jugador.  

\textbf{Precondición:} El número de jugadores de la sala es al menos el mínimo requerido por la configuración de partida.

\textbf{Escenario exitoso principal:}
\begin{enumerate}[label=\arabic*.]
  \item El jugador presiona en el botón iniciar partida
  \item El sistema pregunta si está seguro de iniciar la partida.
  \item El jugador confirma la acción.
  \item El sistema asigna los turnos y las cartas a los jugadores de acuerdo a las reglas.
\end{enumerate}

\textbf{Escenario alternativo:}
\begin{enumerate}[label=\arabic*.]
  \item[3a.] El jugador no confirma el inicio de partida.
  \item[4a.] El sistema reinicia al estado original antes del caso de uso.
\end{enumerate}
\end{casodeuso}





\begin{casodeuso}{9}{Jugar turno completo}
\textbf{Actor Primario:} Jugador.  

\textbf{Precondición:} El turno actual debe ser del jugador y el temporizador en su máxima capacidad.  

\textbf{Escenario exitoso principal:}
\begin{enumerate}[label=\arabic*.]
  \item El jugador deja pasar el turno hasta que se agote el temporizador.
  \item El sistema descarta una carta de la mano del jugador, levanta otra de la pila principal del mazo y pasa de turno.
\end{enumerate}

\textbf{Escenarios alternativos:}
\begin{enumerate}[label=\arabic*.]
  \item[1a.] El jugador juega una carta de evento.  
  \item[2a.] El sistema aplica el efecto de la carta y la descarta.  
  \item[3a.] El jugador elige si descarta más cartas de su mano.
  \item[4a.] El sistema envía las cartas a la pila de descarte.
  \item[5a.] El jugador elige de dónde reponer sus cartas hasta que tenga seis en mano.
  \item[6a.] El sistema le asigna sus cartas y pasa de turno.
  \vspace{1.5em} % un reglón y medio de espacio 
  \item[1b.] El jugador juega un set de detectives completo.  
  \item[2b.] El sistema asigna el set al frente del jugador y aplica los efectos del set de detective.  
  \item[3b.] El jugador elige si descarta más cartas de su mano.
  \item[4b.] El sistema envía las cartas a la pila de descarte.
  \item[5b.] El jugador elige de dónde reponer sus cartas hasta que tenga seis en mano.
  \item[6b.] El sistema le asigna sus cartas y pasa de turno.
  \vspace{1.5em} 
  \item[1c.] El jugador juega una carta de detectives de un set ya existente.  
  \item[2c.] El sistema aplica los efectos del set de detective.  
  \item[3c.] El jugador elige si descarta más cartas de su mano.
  \item[4c.] El sistema envía las cartas a la pila de descarte.
  \item[5c.] El jugador elige de dónde reponer sus cartas hasta que tenga seis en mano.
  \item[6c.] El sistema le asigna sus cartas y pasa de turno.
  \vspace{1.5em} 
  \item[1d.] El jugador descarta X número de cartas.  
  \item[2d.] El sistema las envía a la pila de descarte. 
  \item[3d.] El jugador elige de dónde levantar X cartas. 
  \item[4d.] El sistema concluye su turno. 
\end{enumerate}

\newpage

\textbf{Escenario excepcional:}
\begin{enumerate}[label=\arabic*.]
  \item[1aa.] (Aplica a los escenarios alternativos 1a, 1b y 1c) Algún otro jugador juega una carta `Not So Fast'.
  
  \textbf{--} Caso de uso de `Not So Fast'.
\end{enumerate}

\end{casodeuso}




\begin{casodeuso}{14}{Jugar carta \textit{`Not So Fast'}}
\textbf{Actor Primario:} Jugador.  
\textbf{Actor Secundario:} Otro jugador en su turno.  

\textbf{Precondición:} El jugador debe tener al menos una carta de `Not So Fast' en su mano y otro jugador debe haber usado una carta con algún tipo de efecto.  

\textbf{Escenario exitoso principal:}
\begin{enumerate}[label=\arabic*.]
  \item El jugador juega su carta de `Not So Fast'.
  \item El sistema cancela la acción del efecto de la carta jugada más reciente.
\end{enumerate}

\textbf{Escenario excepcional:}
\begin{enumerate}[label=\arabic*.]
  \item[1aa.] Algún otro jugador juega una carta `Not So Fast'. \\
  \textbf{--} Caso de uso de `Not So Fast'.
\end{enumerate}

% Aquí seguirías con escenarios alternativos y excepcionales...
\end{casodeuso}
\vspace{2cm}
\begin{flushright}
\textbf{Emiliano Guardabassi}
\end{flushright}

\end{document}
