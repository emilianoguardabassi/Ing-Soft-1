\documentclass[12pt,a4paper]{article}

\usepackage[utf8]{inputenc}
\usepackage[T1]{fontenc}
\usepackage[spanish]{babel}
\usepackage{enumitem}   % para personalizar listas

\title{\textbf{Catálogo de Casos de Uso}}
\author{Emiliano Guardabassi}
\date{}

\begin{document}

\maketitle

\section*{Catálogo de Casos de Uso}

\begin{enumerate}[
  label=\textbf{Caso de Uso \#\arabic*:},
  labelsep=0.6em,   % espacio entre etiqueta y texto
  labelwidth=3.5cm, % ancho fijo para TODAS las etiquetas
  leftmargin=!,     % calcula el margen con labelwidth+labelsep
  align=left        % alinear etiquetas a la izquierda
]
  \item Ingresar datos de jugador
  \item Crear partida
  \item Unirse a partida
  \item Iniciar partida
  \item Escribir en el chat
  \item Leer el chat
  \item Agotar temporizador.
  \item Asignar temporizador.
  \item Jugar turno completo
  \item Jugar carta de detective
  \item Bajar set de detective
  % \item Jugar set \textit{`Detective Hercule Poirot'}
  % \item Jugar set \textit{`Detective Miss Marple'}
  % \item Jugar set \textit{`Detective Mr Satterhwaite'}
  % \item Jugar set \textit{`Detective Parker Pyne'}
  % \item Jugar set \textit{`Detective Lady Eileen `Bundle' Brent'}
  % \item Jugar set \textit{`Detective Tommy Beresford'}
  % \item Jugar set \textit{`Detective Tuppence Beresford'}
  % \item Jugar set mixto \textit{`Detectives Beresford'}
  % \item Jugar set mixto con \textit{`Detective Harley Quin WildCard'}
  % \item Jugar \textit{`Detective Hercule Poirot'} a set existente
  % \item Jugar \textit{`Detective Miss Marple'} a set existente
  % \item Jugar \textit{`Detective Mr Satterhwaite'} a set existente
  % \item Jugar \textit{`Detective Parker Pyne'} a set existente
  % \item Jugar \textit{`Detective Lady Eileen `Bundle' Brent'} a set existente
  % \item Jugar \textit{`Detective Tommy Beresford'} a set existente
  % \item Jugar \textit{`Detective Tuppence Beresford'} a set existente
  % \item Jugar \textit{`Detective Ariadne Oliver'} a set existente
  \item Jugar carta de evento
  % \item Jugar carta \textit{`Cards on the table'}
  % \item Jugar carta \textit{`Another victim'}
  % \item Jugar carta \textit{`Dead Card Folly'}
  % \item Jugar carta \textit{`Look Into The Ashes'}
  % \item Jugar carta \textit{`Card Trade'}
  % \item Jugar carta \textit{`And Then There Was One More'}
  % \item Jugar carta \textit{`Delay The Murderer's Escape'}
  % \item Jugar carta \textit{`Early Train To Paddington'}
  % \item Jugar carta \textit{`Point Your Suspicions'}
  \item Recibir carta del tipo `Devious card'
  % \item Recibir carta \textit{`Blackmailed'}
  % \item Recibir carta \textit{`Social Faux Pas'}
  \item Jugar carta \textit{`Not So Fast'}
  \item Tomar carta del mazo.
  \item Descartar carta.
  \item Elegir un jugador.
  \item Elegir un secreto ya revelado.
  \item Revelar un secreto propio a elección.
  \item Revelar secreto de otro jugador a tu elección.
  \item Revelar secreto de otro jugador a su elección.
  \item Añadir un secreto ajeno revelado a tus secretos sin revelar.
  \item Votar a un jugador.
  \item Robar un set de detectives.
  \item Decidir dirección (derecha o izquierda).
  \item Dar una carta a un jugador.
  \item Indicar carta por su nombre.
  \item Examinar cartas de la pila de descarte.
  \item Tomar carta de la pila de descarte.
  \item Cambiar una carta con otro jugador.
  \item Esconder un secreto ya revelado.
  \item Añadir carta al mazo.
  \item Mostrar un secreto propio a otro jugador.
\end{enumerate}



\vspace{2cm}
\begin{flushright}
\textbf{Emiliano Guardabassi}
\end{flushright}

\end{document}